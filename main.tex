\def\solution{2} % 0: Klausurvariante, 1: Sch\"ulerversion, 2: Betreuerversion

\def\scannable{1} % 0 oder 1 für false oder true
\def\wettbewerb{icho} %cds oder icho
\def\klasse{10}
\def\bundesland{Mecklenburg-Vorpommern}
\def\cyear{2025}
\def\round{3} % 1, 2, 3, 4
\def\exam{1} % 1, 2; Nur fuer 3. & 4. Runde
\def\fullname{57. IChO 2025 in den Vereinigten Arabischen Emiraten} %z.B. 56. IChO 2024 in Saudi-Arabien (fuer Deckblatt)
\def\code{0} %0: Linie; 1: aus Datei importieren
%code 0 für leeren Schülercode -> selbst eintragen; code 1 für Serienbrief (sowohl cds als auch icho)
\def\kastenanzeigen{1} %0: in der Klausurvariante ohne Kasten (keine Auswirkung auf die anderen Varianten), 1: immer Kasten anzeigen
\def\paperoption{plainpaper}
\documentclass[\paperoption]{ichobooklet}

\usepackage{todonotes} %zum Ausblenden [disable] % Für Aufgabenentwicklungsprozess nicht für finales Ergebnis


% das Fehlen eines Schülercodes wirft Fehler, wenn Dokumente einzeln kompiliert werden, deswegen hier definiert, um dies zu umgehen
% das sollte keinen Einfluss auf Serienbriefe haben (bitte trd vor dem Drucken checken)
% optional können auch susbl, susfirstname und suslastname so vordefiniert werden, falls diese Fehler werfen

\def\suscode{KEIN SCHÜLERCODE!}\opt{icho}{\opt{rd2}{\def\susbl{}}}%tvgdvijefiourwijo


\begin{document}

%serienbriefstart 
\newread\quelle
\openin\quelle=\opt{c1}{Codes.csv}\opt{c0}{./Hilfsdateien/Dummycode.csv}
\loop\read\quelle to \zeile
  \ifeof\quelle\global\morefalse
  \else\expandafter\chopline\zeile\\

\setcounter{page}{1}\setcounter{section}{0}
% resettet counter, z.b. equation numbering

\subfile{./Hilfsdateien/deckblatt}
\subfile{./Hilfsdateien/formelsammlung}
\subfile{./Hilfsdateien/periodensystem}
\subfile{./Hilfsdateien/NMR}

%%%%%%%%%%%%%%%%%%
\subfile{Dokumentation/Beispielaufgabe}
\subfile{To-do}

%serienbriefende
  \fi
\ifmore\repeat
\closein\quelle
\end{document}
