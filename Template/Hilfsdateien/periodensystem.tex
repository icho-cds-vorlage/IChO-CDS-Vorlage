\documentclass[../main.tex]{subfiles}
\graphicspath{{\subfix{../Abbildungen/}}}
\begin{document}
% Draw the Contour of Groups
\def\DrawGroup;{
    \clip[]
        (16in,0)|-+(-1in,1in)|-+(2in,-6in)|-cycle
        (17in,-2in)rectangle+(10in,-4in)
        (17in,-7in)rectangle+(14in,-2in)
        (32in,-0in)-|+(-5in,-6in)-|+(1in,1in)-|cycle;
}

% Atomic Number \PNn => Coordinate (\PNx,\PNy)
\def\PositionByNumber;{
    \pgfmathtruncatemacro\PNn{\n}
    \foreach\PNa/\PNb in{1/30,34/24,66/24,98/14,130/14}{
        \ifnum\PNn>\PNa
            \pgfmathtruncatemacro\PNn{\PNn+\PNb}\xdef\PNn{\PNn}
        \fi
    }
}

% Coordinate (\PNx,\PNy) => Move and Group
\def\GroupByPosition;{
    \pgfmathtruncatemacro\PNx{mod(\PNn-0.5,32)+.5}
    \pgfmathtruncatemacro\PNy{(\PNn-\PNx)/32}
    \ifnum\PNx>2\ifnum\PNx<17
        \pgfmathtruncatemacro\PNy{\PNy+3}
        \pgfmathtruncatemacro\PNx{\PNx+14}
    \fi\fi
    \ifnum\PNx<3
        \pgfmathtruncatemacro\PNx{\PNx+14}
    \fi
    \ifnum\PNx>26\ifnum\PNy<8
        \pgfmathtruncatemacro\PNx{\PNx}
    \fi\fi
    \pgftransformreset\pgftransformxshift{\PNx in}\pgftransformyshift{-\PNy in}
}

\newread\linereader
\def\foreachline#1#2{\openin\linereader=#1\nextline#2}
\def\nextline#1{\read\linereader to \line\ifeof\linereader\closein\linereader\else\expandafter#1\line\end\expandafter\nextline\expandafter#1\fi}
\def\element#1 #2 #3 #4\end{
    \xdef\n{#1}
    \PositionByNumber;
    \GroupByPosition;
    \draw[black](0,0)rectangle(1in,1in)
        (.5,.9in)node{\Large #1}
        (.5in,.1in)node{\Large #4}
        (.5in,.5in)node{\Huge #2};
}

\begin{center}
    \rotatebox{90}{\resizebox{\textheight}{!}{
        \begin{tikzpicture}
            \DrawGroup;
            \foreachline{./Hilfsdateien/elements.txt}\element
        \end{tikzpicture}
    }}
\end{center}
\newpage
\end{document}